\documentclass[10pt, oneside, letterpaper]{article}
\usepackage[utf8]{inputenc}

\usepackage{authblk}
	\setcounter{Maxaffil}{0}
	\renewcommand\Affilfont{\itshape\small}
\usepackage[margin=1.3in]{geometry}
\usepackage{graphicx}
	\graphicspath{ {img/} }
\usepackage[colorlinks=true, linkcolor=blue, citecolor=blue]{hyperref}
\usepackage{indentfirst}
	\setlength{\parindent}{15pt}
\usepackage{multicol}
\usepackage[raggedright]{titlesec}
\usepackage{verbatim}

% Second set of package loads
\usepackage{apacite}  % Must occur after hyperref

\title{\textbf{Building the Tower of Babel:\\Ethical Considerations of Human Enhancement}}
\author{Jeffrey Leung}
\affil{Simon Fraser University}
\date{November 19, 2018}

\begin{document}

	\maketitle
	
	\begin{abstract}
		This paper is an assignment for CMPT 320: Social Implications of a Computerized Society. This paper discusses the emergent impacts of human enhancement on society. Cybernetic and biological enhancements are analyzed for their technological progress and potential development. Ethical perspectives on the issue are analyzed through the lens of cybernetic bioaugmentations and prosthetics. These ethical perspectives are the deontological, anthropological, rights-based, fairness, and utilitarian viewpoints. The judgements discussed are the righteousness of the actions and end results, and the individual impacts. Through the analyses, we conclude that regulations and security protocols should be revised as enhancement technology advances and that society should drive the innovation of enhancement towards accessibility and collective utility.
	\end{abstract}
	
	\begin{multicols}{2}
	
	\section{Introduction}

	Humans have always sought to design tools to solve problems, to build bridges, and to strike down enemies. As our ability to design has grown increasingly sophisticated, our creations have also become more integrated with our lives. Prosthetics repair and often augment the strength and capabilities of our bodies, vaccines improve our immune system to fight diseases, and neurostimulative drugs such as Ritalin enhance cognitive abilities \cite{MyHealth2017}. Creations are often designed to compensate for a lack of capability or an augmentation of natural abilities. Society is forced to continually reconsider its stance on permitting, regulating, or preventing such technologies altogether.
	
	In this paper, we will discuss the topic of emergent human enhancement by describing its potential developments, social implications, and moral considerations through the use of fundamental ethical frameworks. Mechanical prosthetics attached to the body will be used for contextual examples and theoretical case studies.
	
	\section{Areas and Impacts of Enhancement}
	
	To understand the intrinsic nature and controversy of enhancement, the distinction between enhancement and therapy must first be made. Therapy is defined as the action of returning a human body to its norm of functioning --- for example, a wheelchair for a paraplegic or a vaccine to prevent an infection \cite{Allhoff2009}. Conversely, enhancement is the action of giving a human body abilities beyond what a human would be expected to achieve --- for example, prosthetic legs which can run faster than human legs or anabolic steroids to increase bodybuilding. Therapy is easily justifiable but enhancement inspires ethical debates on the nature of its righteousness \cite{Bostrom2008Eth}.
	
	One key area of enhancement is biological substances and manipulative design. Biologists tweak genetic traits and solve the deterioration of the body due to age. Substances such as steroids and stimulants increase the body's senses and abilities far beyond normal functioning. These senses may be physical, cognitive (e.g. Ritalin) or psychological (e.g. hallucinogens). These substances are at risk of being regulated by patents and companies, which occurs as a core conflict in the game Deus Ex \cite{Eidos2011}.
	
	The other key area of enhancement is the cybernetic enhancement of the body. \citeA{Herr2012} describe how prosthetics today can be designed to provide a natural walking gait, and further work demonstrates how cybernetic limbs can exceed human limbs for tasks such as climbing or ballet. The prosthetic augmentation of humans may soon be at the forefront of warfare and civilian threats. Some technologies seek to be invisible within the body to enhance physical capabilities (e.g. stronger muscles) or cognitive abilities (e.g. neuroprosthetics \cite{Moore2005}).
	
	The potential impacts of enhancements cross societal divides from industry to education to military. When companies begin to use these technologies or when augmented employees are preferred, then imbalances in the free market will manifest. Consumer protection may become an issue that society, employers, and governments must contend to in this context. Education and testing will be challenged by invisible internalized technologies. It is clear that many enhancements exist and will continue to be developed which allow the human body to surpass socially accepted normal levels.
	
	\section{On Deontology, Rights, and Identity}
	
	First we will evaluate human enhancement from the perspective of the inherent action and move towards the speculative resultant state.
	
	Like education, enhancement can be viewed as the fundamental action to improve ourselves. Disregarding the ends for which we improve ourselves, there is nothing intrinsically wrong with this action --- therefore, enhancement is deontologically sound and inherently justifiable.
	
	In the context of education, the act of self-betterment is a positive right. However, other methods of self-betterment are negative rights as we are permitted to use them without interference --- for example, cosmetics and improving health, which can be compared easily to enhancements \cite{Bostrom2008Phuman}. \citeA[p.~628]{Roduit2015} describes allowing individuals to pursue individual ideals within social contexts, as long as they are not detrimental to specific rules.
	
	It is easy to imagine how prosthetics with combat capabilities could be harmful to others. The question will arise as to whether there are significant illegal uses to justify preventing specific enhancements. After all, a tool is a focusing of ``human will toward any feasible object'' \cite{Galvan2011} and that tool may significantly increase the ability to harm. The positive right to improve ourselves may increase our infringement upon the negative rights of others to be safe. It is difficult to predict what devices will be developed, but they will threaten the nature of security such as when dangerous enhancements are nearly undetectable --- for example, neuroprosthetics \cite{Coates2008}. I predict that the government will interject with legal hurdles in reactive fashion such as prohibition and society will work to counter-normalize enhancements when appropriate.
	
	\citeA{Bostrom2008Phuman} posits that a posthuman is just as comparable to other posthumans and as such there is little appreciable value in staying purely human, and \citeA[pp.~103]{Heilinger2014} argues that many aspects of human nature may be natural but not inherently good. In opposition, I postulate that there exists a discernible intrinsic value in a purely human accomplishment, though not always enforceable. Still, I cannot dismiss the substantial value to the self and society in advancing enhancement technology.
	
	\section{On Universalizability, Fairness, and Societal Division}
	
	The result of enhancement is reflected on the greater good - under the veil of ignorance, does the technology increase general well-being? For most developments, people would generally agree. Conversely there will always be Neo-Luddites who oppose technological innovation. The game Deus Ex included a political faction named Purity First which opposed cybernetic enhancement, leading to militaristic activism \cite{Eidos2011}.
	
	Assuming the acceptance of the beneficial nature of enhancement, the question of distributive justice arises - will enhancements be prioritized by need, merit, or another method? It could be argued that imbalance will occur between those who have access and those who do not. Perhaps this division will be caused from high costs, preventing those with lower income from enhancing themselves and being competitive. Another theory is that the divide will come from the skewed accessibility of technology from different locations across the world. Some argue that such a divide would destabilize a democratic society, some believe in individual benefits over societal wellbeing, and others push for the prioritized support of those who require assistance \cite[p.~236]{Guibilini2015}.
	
	Enhancements pose a great danger to society and its unity, as they are yet another tool may make the powerful even more powerful.
	
	\section{On Utilitarianism}
	
	Both the actions and results of enhancement invoke difficult discussions. In the end, the discussion will turn inwards to the self --- does this technology help or hurt me?
	
	The easiest actors to judge are those who are enhanced --- they benefit most personally as they have the will to improve themselves. Those who produce enhancements will profit behind the scenes. Employers of the enhanced will easily benefit from increased productivity or abilities on the job.
	
	Conversely, those in poverty will be less able to be enhanced, which lessens their comparative abilities.	Employers of the unenhanced may also lose competitive edge and could begin to discriminate hiring against the unenhanced.
	
	Enhancement for combat purposes is currently under development as a generally untapped method to gain a tactical advantage over the opponent. The movie Jacob's Ladder imagined the terrifying possibility of the military using hallucinogens on their own troops to increase focus and remove inhibitions \cite{Lyne1990}. Of course, such technology would also likely be tools used against civilians for oppression, surveillance, and tracking. Inversely, it is possible that enhancement could also provide civilians with tools for armed rebellion against governments, similar to an arms race within a country.
	
	It is clear that enhancements will be a net positive, but the negatives for specific entities must be considered carefully.
	
	\section{Conclusion}
	
	Though we have attempted to restrict the areas of therapy and enhancement, I argue that it is inevitable that therapy enables enhancement. When technology is sufficiently advanced, people will be able to upgrade standard prosthetics to tools or weapons. As a result, I believe that this brave new world requires the reconstruction of patent regulations and the manufacture of biological substances to increase freedom and remove inhibitions of corporate control, or innovators will compel society to reconsider these by force. I also theorize that the effectiveness and practicality of security requires re-evaluation for when technology is closely integrated with the human body.
	
	Above all, innovation should be driven by accessibility rather than exclusivity. The government and other entities should direct the development of enhancement technology towards improving the well-being of those who require assistance from society, to lessen imbalance \cite{Moore2005}. It is our responsibility to ensure that future technologies are designed for the collective utility.
	
	Are we exceeding our moral boundaries? Are we playing God by changing the nature of humanity? Our consequences will speak for themselves rather than our dramatic overtures \cite{Clarke2016}. The development of human enhancement technology will force us to confront societal ethical, moral, and legal stances and I look forward to the challenge.
	
	\section{Recommended Readings}
	
	\noindent \hangindent=0.5cm
	Allhoff, F., Lin, P., Moor, J., \& Weckert, J. (2009, August). Ethics of human enhancement: 25 questions \& answers. \textit{US National Science Foundation.}
	
	\noindent \hangindent=0.5cm
	Clarke, S., Savulescu, J., Coady, T., Giubilini, A., \& Sanyal, S. (2016). \textit{The ethics of human enhancement: Understanding the debate}. Oxford Scholarship Online. doi: 10.1093/acprof:oso/9780198754855.001.0001
	
	\noindent \hangindent=0.5cm
	Guibilini, A., \& Sanyal, S. (2015). The ethics of human enhancement. \textit{Philosophy Compass}, 10(4), 233–243. doi: 10.1111/phc3.12208
	
	\end{multicols}
	
	\bibliographystyle{apacite}
	\bibliography{ethical-considerations-of-human-enhancement}
	
\end{document}
